\usepackage{bm}

%\usepackage{graphics}
\usepackage{graphicx}
\usepackage{amsmath,amssymb,amsthm}
\usepackage{bm}
%\usepackage{subfigure}

\usepackage{xcolor}
\usepackage{colortbl}

\def\boldred#1{\color{red}\textbf{#1}}

\def\IC{\mathbb{C}}
\def\IF{\mathbb{F}}
\def\II{\mathbb{I}}
\def\IM{\mathbb{M}}
\def\IN{\mathbb{N}}
\def\IP{\mathbb{P}}
\def\IR{\mathbb{R}}
\def\IZ{\mathbb{Z}}

\def\ba{\mathbf{a}}
\def\bb{\mathbf{b}}
\def\bc{\mathbf{c}}
\def\be{\mathbf{e}}
\def\bh{\mathbf{h}}
\def\bi{\mathbf{i}}
\def\bj{\mathbf{j}}
\def\bk{\mathbf{k}}
\def\bn{\mathbf{n}}
\def\bp{\mathbf{p}}
\def\br{\mathbf{r}}
\def\bs{\mathbf{s}}
\def\bu{\mathbf{u}}
\def\bv{\mathbf{v}}
\def\bw{\mathbf{w}}
\def\bx{\mathbf{x}}
\def\by{\mathbf{y}}
\def\bz{\mathbf{z}}

\def\bB{\mathbf{B}}
\def\bD{\mathbf{D}}
\def\bF{\mathbf{F}}
\def\bG{\mathbf{G}}
\def\bN{\mathbf{N}}
\def\bR{\mathbf{R}}
\def\bS{\mathbf{S}}
\def\bT{\mathbf{T}}
\def\b0{\mathbf{0}}

\bmdefine{\bmu}{\bm{\mu}}

\def\A{\mathcal{A}}
\def\B{\mathcal{B}}
\def\C{\mathcal{C}}
\def\D{\mathcal{D}}
\def\E{\mathcal{E}}
\def\F{\mathcal{F}}
\def\G{\mathcal{G}}
\def\I{\mathcal{I}}
\def\L{\mathcal{L}}
\def\M{\mathcal{M}}
\def\P{\mathcal{P}}
\def\R{\mathcal{R}}
\def\S{\mathcal{S}}
\def\T{\mathcal{T}}
\def\U{\mathcal{U}}
\def\V{\mathcal{V}}

\def\nbOne{{\mathchoice {\rm 1\mskip-4mu l} {\rm 1\mskip-4mu l}
{\rm 1\mskip-4.5mu l} {\rm 1\mskip-5mu l}}}

\def\cov{\ensuremath{\mathsf{cov}}}
\def\Var{\ensuremath{\mathsf{Var}\ }}

\def\defword#1{\textbf{#1}\index{#1}}



%%%%%%%%%%%%%%%%%%%%%
%%%%%%%%%%%%%%%%%%%%%
%%
%%
%% NAVIGATION AND SECTIONING
%%
%%
%%%%%%%%%%%%%%%%%%%%%
%%%%%%%%%%%%%%%%%%%%%
\setbeamertemplate{navigation symbols}{}
\setbeamertemplate{footline}
{%
	\quad\insertsection\hfill p. \insertpagenumber\quad\mbox{}\vskip2pt
}
\usecolortheme{orchid}
\setbeamertemplate{theorems}[numbered]

\makeatletter
\newlength\beamerleftmargin
\setlength\beamerleftmargin{\Gm@lmargin}
\makeatother

%%%%%%%%%%%
% To have links to parts in the outline
\makeatletter
\AtBeginPart{%
	\addtocontents{toc}{\protect\beamer@partintoc{\the\c@part}{\beamer@partnameshort}{\the\c@page}}%
}
%% number, shortname, page.
\providecommand\beamer@partintoc[3]{%
	\ifnum\c@tocdepth=-1\relax
	% requesting onlyparts.
	\makebox[6em]{Chapter #1:} \textcolor{green!30!blue}{\hyperlink{#2}{#2}}
	\par
	\fi
}
\define@key{beamertoc}{onlyparts}[]{%
	\c@tocdepth=-1\relax
}
\makeatother%

\newcommand{\nameofthepart}{}
\newcommand{\nupart}[1]%
{   \part{#1}%
	\renewcommand{\nameofthepart}{#1}%
	{
		\setbeamercolor{background canvas}{bg=orange!50}
		\begin{frame}{#1}%\partpage 
			\hypertarget{\nameofthepart}{}\tableofcontents%
		\end{frame}
	}
}

% Beginning of a section
\AtBeginSection[]{
	{
		\setbeamercolor{background canvas}{bg=orange!10}
		\begin{frame}[noframenumbering,plain]
			\tableofcontents[currentsection,hideothersubsections]
		\end{frame}
		\addtocounter{page}{-1}
		%\addtocounter{framenumber}{-1} 
	}
}

% Beginning of a section
\AtBeginSubsection[]{
	{
		\setbeamercolor{background canvas}{bg=orange!10}
		\begin{frame}[noframenumbering,plain]
			\tableofcontents[currentsection, hideothersubsections, sectionstyle=show/show]
		\end{frame}
		\addtocounter{page}{-1}
		%\addtocounter{framenumber}{-1} 
	}
}




%%%%%%%%%%%%%%%%%%%%%
%%%%%%%%%%%%%%%%%%%%%
%%
%%
%% COLOURED ENVIRONMENTS
%%
%%
%%%%%%%%%%%%%%%%%%%%%
%%%%%%%%%%%%%%%%%%%%%
\newtheorem{proposition}[theorem]{Proposition}
\newtheorem{property}[theorem]{Property}
\newtheorem{importantproperty}[theorem]{Property}
\newtheorem{importanttheorem}[theorem]{Theorem}
%\newtheorem{lemma}[theorem]{Lemma}
%%%%%%% 
%% Definitions in yellow boxes
\usepackage{etoolbox}
\setbeamercolor{block title}{use=structure,fg=structure.fg,bg=structure.fg!05!bg}
\setbeamercolor{block body}{parent=normal text,use=block title,bg=block title.bg!20!bg}

\BeforeBeginEnvironment{definition}{%
	\setbeamercolor{block title}{fg=black,bg=yellow!20!white}
	\setbeamercolor{block body}{fg=black, bg=yellow!05!white}
}
\AfterEndEnvironment{definition}{
	\setbeamercolor{block title}{use=structure,fg=structure.fg,bg=structure.fg!20!bg}
	\setbeamercolor{block body}{parent=normal text,use=block title,bg=block title.bg!50!bg, fg=black}
}
\BeforeBeginEnvironment{importanttheorem}{%
	\setbeamercolor{block title}{fg=black,bg=red!20!white}
	\setbeamercolor{block body}{fg=black, bg=red!05!white}
}
\AfterEndEnvironment{importanttheorem}{
	\setbeamercolor{block title}{use=structure,fg=structure.fg,bg=structure.fg!20!bg}
	\setbeamercolor{block body}{parent=normal text,use=block title,bg=block title.bg!50!bg, fg=black}
}
\BeforeBeginEnvironment{importantproperty}{%
	\setbeamercolor{block title}{fg=black,bg=red!50!white}
	\setbeamercolor{block body}{fg=black, bg=red!30!white}
}
\AfterEndEnvironment{importantproperty}{
	\setbeamercolor{block title}{use=structure,fg=structure.fg,bg=structure.fg!20!bg}
	\setbeamercolor{block body}{parent=normal text,use=block title,bg=block title.bg!50!bg, fg=black}
}


% Colours for special pages
\def\extraContent{yellow!20}

% Allow to change slide colour
% From: https://tex.stackexchange.com/questions/8043/change-the-background-color-of-a-frame-in-beamer
\defbeamertemplate*{background canvas}{mydefault}{%
	\ifbeamercolorempty[bg]{background canvas}{}{\color{bg}\vrule width\paperwidth height\paperheight}% copied beamer default here
}
\defbeamertemplate*{background canvas}{bg}{%
	\color{lightgray!20}\vrule width\paperwidth height\paperheight% added bg color
}
\BeforeBeginEnvironment{frame}{%
	\setbeamertemplate{background canvas}[mydefault]%
}
\makeatletter
\define@key{beamerframe}{bg}[true]{%
	\setbeamertemplate{background canvas}[bg]%
}
\makeatother
% Use with
%\begin{frame}
% \frametitle{Normal}
%\end{frame} 
%\begin{frame}[bg]
% \frametitle{With bg}
%\end{frame}


% Vertical alignment on pages
% From: https://tex.stackexchange.com/questions/148365/how-do-i-ask-beamer-to-exactly-fill-up-a-slide
% Turn on with
% \stretchon
% (outside slide), and off with
% \stretchoff
\def\itemsymbol{$\blacktriangleright$}
%\def\itemsymbol{}
\let\svpar\par
\let\svitemize\itemize
\let\svenditemize\enditemize
\let\svitem\item
\let\svcenter\center
\let\svendcenter\endcenter
\let\svcolumn\column
\let\svendcolumn\endcolumn
\def\newitem{\renewcommand\item[1][\itemsymbol]{\vfill\svitem[##1]}}%
\def\newpar{\def\par{\svpar\vfill}}%
\newcommand\stretchon{%
	\newpar%
	\renewcommand\item[1][\itemsymbol]{\svitem[##1]\newitem}%
	\renewenvironment{itemize}%
	{\svitemize}{\svenditemize\newpar\par}%
	\renewenvironment{center}%
	{\svcenter\newpar}{\svendcenter\newpar}%
	\renewenvironment{column}[2]%
	{\svcolumn{##1}\setlength{\parskip}{\columnskip}##2}%
	{\svendcolumn\vspace{\columnskip}}%
}
\newcommand\stretchoff{%
	\let\par\svpar%
	\let\item\svitem%
	\let\itemize\svitemize%
	\let\enditemize\svenditemize%
	\let\center\svcenter%
	\let\endcenter\svendcenter%
	\let\column\svcolumn%
	\let\endcolumn\svendcolumn%
}
\newlength\columnskip
\columnskip 0pt

%%%%%%%%%%%%%%%%%
\usepackage{tikz}
\usetikzlibrary{shapes,arrows}
\usetikzlibrary{positioning}
\tikzstyle{cloud} = [draw, ellipse,fill=red!20, node distance=0.87cm,
minimum height=2em]
\tikzstyle{line} = [draw, -latex']
\usetikzlibrary{shapes.symbols,shapes.callouts,patterns}
\usetikzlibrary{calc,fit}
\usetikzlibrary{backgrounds}

\usetikzlibrary{decorations.pathmorphing,backgrounds,positioning,fit,petri}
\usetikzlibrary{automata}
