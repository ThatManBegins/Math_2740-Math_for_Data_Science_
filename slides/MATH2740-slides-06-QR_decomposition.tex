\documentclass{beamer}
\usetheme{default}

\def\IC{\mathbb{C}}
\def\IF{\mathbb{F}}
\def\II{\mathbb{I}}
\def\IM{\mathbb{M}}
\def\IN{\mathbb{N}}
\def\IP{\mathbb{P}}
\def\IR{\mathbb{R}}
\def\IZ{\mathbb{Z}}

\def\ba{\mathbf{a}}
\def\bb{\mathbf{b}}
\def\bc{\mathbf{c}}
\def\be{\mathbf{e}}
\def\bh{\mathbf{h}}
\def\bi{\mathbf{i}}
\def\bj{\mathbf{j}}
\def\bk{\mathbf{k}}
\def\bn{\mathbf{n}}
\def\bp{\mathbf{p}}
\def\br{\mathbf{r}}
\def\bs{\mathbf{s}}
\def\bu{\mathbf{u}}
\def\bv{\mathbf{v}}
\def\bw{\mathbf{w}}
\def\bx{\mathbf{x}}
\def\by{\mathbf{y}}
\def\bz{\mathbf{z}}

\def\bB{\mathbf{B}}
\def\bD{\mathbf{D}}
\def\bF{\mathbf{F}}
\def\bG{\mathbf{G}}
\def\bN{\mathbf{N}}
\def\bR{\mathbf{R}}
\def\bS{\mathbf{S}}
\def\bT{\mathbf{T}}
\def\b0{\mathbf{0}}

\def\A{\mathcal{A}}
\def\B{\mathcal{B}}
\def\C{\mathcal{C}}
\def\D{\mathcal{D}}
\def\E{\mathcal{E}}
\def\F{\mathcal{F}}
\def\G{\mathcal{G}}
\def\I{\mathcal{I}}
\def\L{\mathcal{L}}
\def\M{\mathcal{M}}
\def\P{\mathcal{P}}
\def\R{\mathcal{R}}
\def\S{\mathcal{S}}
\def\T{\mathcal{T}}
\def\U{\mathcal{U}}
\def\V{\mathcal{V}}

\newtheorem{proposition}[theorem]{Proposition}
\newtheorem{property}[theorem]{Property}
\newtheorem{importantproperty}[theorem]{Property}
\newtheorem{importanttheorem}[theorem]{Theorem}
%\newtheorem{lemma}[theorem]{Lemma}



\setbeamertemplate{navigation symbols}{}
\setbeamertemplate{footline}
{%
	\quad\insertsection\hfill p. \insertpagenumber\quad\mbox{}\vskip2pt
}
\usecolortheme{orchid}
\setbeamertemplate{theorems}[numbered]

%%%%%%% 
%% Definitions in yellow boxes
\usepackage{etoolbox}
\setbeamercolor{block title}{use=structure,fg=structure.fg,bg=structure.fg!20!bg}
\setbeamercolor{block body}{parent=normal text,use=block title,bg=block title.bg!50!bg}

\BeforeBeginEnvironment{definition}{%
	\setbeamercolor{block title}{fg=black,bg=yellow!50!white}
	\setbeamercolor{block body}{fg=black, bg=yellow!30!white}
}
\AfterEndEnvironment{definition}{
	\setbeamercolor{block title}{use=structure,fg=structure.fg,bg=structure.fg!20!bg}
	\setbeamercolor{block body}{parent=normal text,use=block title,bg=block title.bg!50!bg, fg=black}
}
\BeforeBeginEnvironment{importanttheorem}{%
	\setbeamercolor{block title}{fg=black,bg=red!50!white}
	\setbeamercolor{block body}{fg=black, bg=red!30!white}
}
\AfterEndEnvironment{importanttheorem}{
	\setbeamercolor{block title}{use=structure,fg=structure.fg,bg=structure.fg!20!bg}
	\setbeamercolor{block body}{parent=normal text,use=block title,bg=block title.bg!50!bg, fg=black}
}
\BeforeBeginEnvironment{importantproperty}{%
	\setbeamercolor{block title}{fg=black,bg=red!50!white}
	\setbeamercolor{block body}{fg=black, bg=red!30!white}
}
\AfterEndEnvironment{importantproperty}{
	\setbeamercolor{block title}{use=structure,fg=structure.fg,bg=structure.fg!20!bg}
	\setbeamercolor{block body}{parent=normal text,use=block title,bg=block title.bg!50!bg, fg=black}
}



\title{QR factorisation and application to least squares}
\date{}

\begin{document}
\DeclareFontShape{OT1}{cmss}{b}{n}{<->ssub * cmss/bx/n}{} 
\begin{frame}
	\titlepage
\end{frame}


%%%%%%%%%%%%%%%%%%%%%
%%%%%%%%%%%%%%%%%%%%%
%%%%%%%%%%%%%%%%%%%%%
%%%%%%%%%%%%%%%%%%%%%


\begin{frame}{Matrix factorisations}
Matrix factorisations are popular because they allow to perform some computations more easily
\vfill
There are several different types of factorisations. Here, we study just the QR factorisation, which is useful for many least squares problems
\vfill
Before we start, though, we need to learn a little about \emph{orthogonal sets} and \emph{orthogonal matrices}
\end{frame}


\begin{frame}
\begin{definition}[Orthogonal set of vectors]
The set of vectors $\{\bv_1,\ldots,\bv_k\}\in\IR^n$ is an \textbf{orthogonal set} if
\[
\forall i,j=1,\ldots,k,\quad i\neq j \implies \bv_i\bullet\bv_j=0
\]
\end{definition}

\begin{theorem}\label{th:ortho_implies_LI}
$\{\bv_1,\ldots,\bv_k\}\in\IR^n$ with $\forall i$, $\bv_i\neq\b0$, orthogonal set $\implies$ $\{\bv_1,\ldots,\bv_k\}\in\IR^n$ linearly independent
\end{theorem}

\begin{definition}[Orthogonal basis]
Let $S$ be a basis of the subspace $W\subset\IR^n$ composed of an orthogonal set of vectors. We say $S$ is an \textbf{orthogonal basis} of $W$
\end{definition}
\end{frame}

\begin{frame}{Proof of Theorem~\ref{th:ortho_implies_LI}}
Assume $\{\bv_1,\ldots,\bv_k\}$ orthogonal set with $\bv_i\neq\b0$ for all $i=1,\ldots,k$. Recall $\{\bv_1,\ldots,\bv_k\}$ is LI if 
\[
c_1\bv_1+\cdots+c_k\bv_k=\b0\iff c_1=\cdots=c_k=0
\]
So assume $c_1,\ldots,c_k\in\IR$ are s.t. $c_1\bv_1+\cdots+c_k\bv_k=\b0$.
Recall that $\forall\bx\in\IR^k$, $\b0_k\bullet\bx=0$. So for some $\bv_i\in\{\bv_1,\ldots,\bv_k\}$
\begin{align}
0 &= \b0\bullet\bv_i \nonumber \\
&= (c_1\bv_1+\cdots+c_k\bv_k)\bullet\bv_i \nonumber \\
&= c_1\bv_1\bullet\bv_i+\cdots+c_k\bv_k\bullet\bv_i \label{eq:proof_th_ortho_implies_LI}
\end{align}
As $\{\bv_1,\ldots,\bv_k\}$ orthogonal, $\bv_j\bullet\bv_i=0$ when $i\neq j$, \eqref{eq:proof_th_ortho_implies_LI} reduces to
\[
c_i\bv_i\bullet\bv_i = 0 \iff c_i\|\bv_i\|^2 = 0
\]
As $\bv_i\neq 0$ for all $i$, $\|\bv_i\|\neq 0$ and so $c_i=0$. This is true for all $i$, hence the result
\end{frame}

\begin{frame}{Example -- Vectors of the standard basis of $\IR^3$}
For $\IR^3$, we denote
\[
\bi =\begin{pmatrix}
1\\0\\0
\end{pmatrix},\quad
\bj =\begin{pmatrix}
0\\1\\0
\end{pmatrix}
\textrm{ and }
\bk =\begin{pmatrix}
0\\0\\1
\end{pmatrix}
\]
($\IR^k$ for $k>3$, we denote them $\be_i$)
\vfill
Clearly, $\{\bi,\bj\}$, $\{\bi,\bk\}$, $\{\bj,\bk\}$ and $\{\bi,\bj,\bk\}$ orthogonal sets. The standard basis vectors are also $\neq\b0$, so the sets are LI. And
\[
\{\bi,\bj,\bk\}
\]
is an orthogonal basis of $\IR^3$ since it spans $\IR^3$ and is LI
\vfill
\[
c_1\bi+c_2\bj+c_3\bk
=
c_1\begin{pmatrix}
1\\0\\0
\end{pmatrix}
+c_2\begin{pmatrix}
0\\1\\0
\end{pmatrix}
+c_3\begin{pmatrix}
0\\0\\1
\end{pmatrix}
=
\begin{pmatrix}
c_1\\c_2\\c_3
\end{pmatrix}
\]
\end{frame}

\begin{frame}{Orthonormal version of things}
\begin{definition}[Orthonormal set]
The set of vectors $\{\bv_1,\ldots,\bv_k\}\in\IR^n$ is an \textbf{orthonormal set} if it is an orthogonal set and furthermore
\[
\forall i=1,\ldots,k,\quad \|\bv_i\|=1
\]
\end{definition}
\begin{definition}[Orthonormal basis]
A basis of the subspace $W\subset\IR^n$ is an \textbf{orthonormal basis} if the vectors composing it are an orthonormal set
\end{definition}
\vfill
$\{\bv_1,\ldots,\bv_k\}\in\IR^n$ is orthonormal if
\[
\bv_i\bullet\bv_j =
\begin{cases}
1 &\textrm{if }i=j \\
0 &\textrm{otherwise}
\end{cases}
\]
\end{frame}


\begin{frame}
\begin{theorem}
Let $Q\in\M_{mn}$. The columns of $Q$ form an orthonormal set if and only if
\[
Q^TQ=\II_n
\]
\end{theorem}
\begin{definition}[Orthogonal matrix]
$Q\in\M_n$ is an \textbf{orthogonal matrix} if its columns form an orthonormal set 
\end{definition}
So $Q\in\M_n$ orthogonal if $Q^TQ=\II$, i.e., $Q^T=Q^{-1}$
\begin{theorem}[NSC for orthogonality]
$Q\in\M_n$ orthogonal $\iff$ $Q^{-1} = Q^T$
\end{theorem}
\end{frame}


\begin{frame}
\begin{theorem}[Orthogonal matrices ``encode" isometries]
\label{th:TFAE_orthogonal_matrices}
Let $Q\in\M_n$. TFAE
\begin{enumerate}
\item $Q$ orthogonal
\item $\forall\bx\in\IR^n$, $\|Q\bx\|=\|\bx\|$
\item $\forall\bx,\by\in\IR^n$, $Q\bx\bullet Q\by=\bx\bullet\by$
\end{enumerate}
\end{theorem}
\vfill
\begin{theorem}\label{th:properties_orthogonal_matrices}
Let $Q\in\M_n$ be orthogonal. Then
\begin{enumerate}
\item The rows of $Q$ form an orthonormal set
\item $Q^{-1}$ orthogonal
\item $\det Q=\pm 1$
\item $\forall\lambda\in\sigma(Q)$, $|\lambda|=1$
\item If $Q_2\in\M_n$ also orthogonal, then $QQ_2$ orthogonal
\end{enumerate}
\end{theorem}
\end{frame}



\begin{frame}{Proof of 4 in Theorem~\ref{th:properties_orthogonal_matrices}}
All statements in Theorem~\ref{th:properties_orthogonal_matrices} are easy, but let's focus on 4 
\vfill
Let $\lambda$ be an eigenvalue of $Q\in\M_n$ orthogonal, i.e., $\exists\IR^n\ni\bx\neq\b0$ s.t.
\[
Q\bx = \lambda\bx
\]
Take the norm on both sides
\[
\|Q\bx\| = \|\lambda\bx\|
\]
From 2 in Theorem~\ref{th:TFAE_orthogonal_matrices}, $\|Q\bx\|=\|\bx\|$ and from the properties of norms, $\|\lambda\bx\|=|\lambda|\;\|\bx\|$, so we have
\[
\|Q\bx\| = \|\lambda\bx\| \iff \|\bx\| = |\lambda|\;\|\bx\| \iff 1=|\lambda|
\]
(we can divide by $\|\bx\|$ since $\bx\neq \b0$ as an eigenvector)
\end{frame}


\begin{frame}{Projections}
\begin{definition}[Orthogonal projection onto a subspace]
$W\subset\IR^n$ a subspace and $\{\bu_1,\ldots,\bu_k\}$ an orthogonal basis of $W$. $\forall\bv\in\IR^n$, the \textbf{orthogonal projection} of $\bv$ \textbf{onto} $W$ is
\[
\mathsf{proj}_W(\bv) =
\frac{\bu_1\bullet\bv}{\|\bu_1\|^2}\bu_1
+\cdots+
\frac{\bu_k\bullet\bv}{\|\bu_k\|^2}\bu_k
\]  
\end{definition}
\vfill
\begin{definition}[Component orthogonal to a subspace]
$W\subset\IR^n$ a subspace and $\{\bu_1,\ldots,\bu_k\}$ an orthogonal basis of $W$. $\forall\bv\in\IR^n$, the \textbf{component} of  $\bv$ \textbf{orthogonal to} W is
\[
\mathsf{perp}_W(\bv)=\bv-\mathsf{proj}_W(\bv)
\]
\end{definition}	
\end{frame}


\begin{frame}
We could spend a long time on these interesting notions, but let us get back to the QR decomposition
\vfill
What this aims to do is to construct an orthogonal basis for a subspace $W\subset\IR^n$
\vfill
To do this, we use the \emph{Gram-Schmidt orthogonalisation process}, which turn s a basis of $W$ into an orthogonal basis of $W$
\end{frame}

\begin{frame}{Gram-Schmidt process}
\begin{theorem}
$W\subset\IR^n$ a subset and $\{\bx_1,\ldots,\bx_k\}$ a basis of $W$. Let
\begin{align*}
\bv_1 &= \bx_1 \\
\bv_2 &= \bx_2 -\frac{\bv_1\bullet\bx_2}{\|\bv_1\|^2}\bv_1 \\
\bv_3 &= \bx_3 -\frac{\bv_1\bullet\bx_3}{\|\bv_1\|^2}\bv_1 -\frac{\bv_2\bullet\bx_3}{\|\bv_2\|^2}\bv_2 \\
&\;\;\vdots & \\
\bv_k &= \bx_k -\frac{\bv_1\bullet\bx_k}{\|\bv_1\|^2}\bv_1 -\cdots-\frac{\bv_{k-1}\bullet\bx_k}{\|\bv_{k-1}\|^2}\bv_{k-1}
\end{align*}
and
\[
W_1=\mathsf{span}(\bx_1),W_2 = \mathsf{span}(\bx_1,\bx_2),\ldots,
W_k = \mathsf{span}(\bx_1,\ldots,\bx_k)
\]
Then $\forall i=1,\ldots,k$, $\{\bv_1,\ldots,\bv_i\}$ orthogonal basis for $W_i$
\end{theorem}
\end{frame}


\begin{frame}{The QR factorisation}
\begin{theorem}\label{th:QR_factorisation}
Let $A\in\M_{mn}$ with LI columns. Then $A$ can be factored as
\[
A=QR
\]
where $Q\in\M_{mn}$ has orthonormal columns and $R\in\M_n$ is nonsingular upper triangular
\end{theorem}
\end{frame}


\begin{frame}{Back to least squares}
So what was the point of all that..?
\vfill
\begin{theorem}[Least squares with QR factorisation]
\label{th:LSQ_with_QR}
$A\in\M_{mn}$ with LI columns, $\bb\in\IR^m$. If $A=QR$ is a QR factorisation of $A$, then the unique least squares solution $\tilde\bx$ of $A\bx=\bb$ is
\[
\tilde\bx = R^{-1}Q^T\bb
\]
\end{theorem}
\end{frame}


\begin{frame}{Proof of Theorem~\ref{th:LSQ_with_QR}}
$A$ has LI columns so 
\begin{itemize}
\item least squares $A\bx=\bb$ has unique solution $\tilde\bx=(A^TA)^{-1}A^T\bb$
\item by Theorem~\ref{th:QR_factorisation}, $A$ can be written as $A=QR$ with $Q\in\M_{mn}$ with orthonormal columns and $R\in\M_n$ nonsingular and upper triangular
\end{itemize}
So
\begin{align*}
A^TA\tilde\bx= A^T\bb &\implies (QR)^TQR\tilde\bx = (QR)^T\bb \\
&\implies R^TQ^TQR\tilde\bx = R^TQ^T\bb \\
&\implies R^T\II_nR\tilde\bx = R^TQ^T\bb \\
&\implies R^TR\tilde\bx = R^TQ^T\bb \\
&\implies (R^T)^{-1}R\tilde\bx = (R^T)^{-1}R^TQ^T\bb \\
&\implies R\tilde\bx = Q^T\bb \\
&\implies \tilde\bx = R^{-1}Q^T\bb 
\end{align*}
\end{frame}

\end{document}