\documentclass{beamer}
\usetheme{default}

\def\IC{\mathbb{C}}
\def\IF{\mathbb{F}}
\def\II{\mathbb{I}}
\def\IM{\mathbb{M}}
\def\IN{\mathbb{N}}
\def\IP{\mathbb{P}}
\def\IR{\mathbb{R}}
\def\IZ{\mathbb{Z}}

\def\ba{\mathbf{a}}
\def\bb{\mathbf{b}}
\def\bc{\mathbf{c}}
\def\be{\mathbf{e}}
\def\bh{\mathbf{h}}
\def\bi{\mathbf{i}}
\def\bj{\mathbf{j}}
\def\bk{\mathbf{k}}
\def\bn{\mathbf{n}}
\def\bp{\mathbf{p}}
\def\br{\mathbf{r}}
\def\bs{\mathbf{s}}
\def\bu{\mathbf{u}}
\def\bv{\mathbf{v}}
\def\bw{\mathbf{w}}
\def\bx{\mathbf{x}}
\def\by{\mathbf{y}}
\def\bz{\mathbf{z}}

\def\bB{\mathbf{B}}
\def\bD{\mathbf{D}}
\def\bF{\mathbf{F}}
\def\bG{\mathbf{G}}
\def\bN{\mathbf{N}}
\def\bR{\mathbf{R}}
\def\bS{\mathbf{S}}
\def\bT{\mathbf{T}}
\def\b0{\mathbf{0}}

\def\A{\mathcal{A}}
\def\B{\mathcal{B}}
\def\C{\mathcal{C}}
\def\D{\mathcal{D}}
\def\E{\mathcal{E}}
\def\F{\mathcal{F}}
\def\G{\mathcal{G}}
\def\I{\mathcal{I}}
\def\L{\mathcal{L}}
\def\M{\mathcal{M}}
\def\P{\mathcal{P}}
\def\R{\mathcal{R}}
\def\S{\mathcal{S}}
\def\T{\mathcal{T}}
\def\U{\mathcal{U}}
\def\V{\mathcal{V}}

\newtheorem{proposition}[theorem]{Proposition}
\newtheorem{property}[theorem]{Property}
\newtheorem{importantproperty}[theorem]{Property}
\newtheorem{importanttheorem}[theorem]{Theorem}
%\newtheorem{lemma}[theorem]{Lemma}



\setbeamertemplate{navigation symbols}{}
\setbeamertemplate{footline}
{%
	\quad\insertsection\hfill p. \insertpagenumber\quad\mbox{}\vskip2pt
}
\usecolortheme{orchid}
\setbeamertemplate{theorems}[numbered]

%%%%%%% 
%% Definitions in yellow boxes
\usepackage{etoolbox}
\setbeamercolor{block title}{use=structure,fg=structure.fg,bg=structure.fg!20!bg}
\setbeamercolor{block body}{parent=normal text,use=block title,bg=block title.bg!50!bg}

\BeforeBeginEnvironment{definition}{%
	\setbeamercolor{block title}{fg=black,bg=yellow!50!white}
	\setbeamercolor{block body}{fg=black, bg=yellow!30!white}
}
\AfterEndEnvironment{definition}{
	\setbeamercolor{block title}{use=structure,fg=structure.fg,bg=structure.fg!20!bg}
	\setbeamercolor{block body}{parent=normal text,use=block title,bg=block title.bg!50!bg, fg=black}
}
\BeforeBeginEnvironment{importanttheorem}{%
	\setbeamercolor{block title}{fg=black,bg=red!50!white}
	\setbeamercolor{block body}{fg=black, bg=red!30!white}
}
\AfterEndEnvironment{importanttheorem}{
	\setbeamercolor{block title}{use=structure,fg=structure.fg,bg=structure.fg!20!bg}
	\setbeamercolor{block body}{parent=normal text,use=block title,bg=block title.bg!50!bg, fg=black}
}
\BeforeBeginEnvironment{importantproperty}{%
	\setbeamercolor{block title}{fg=black,bg=red!50!white}
	\setbeamercolor{block body}{fg=black, bg=red!30!white}
}
\AfterEndEnvironment{importantproperty}{
	\setbeamercolor{block title}{use=structure,fg=structure.fg,bg=structure.fg!20!bg}
	\setbeamercolor{block body}{parent=normal text,use=block title,bg=block title.bg!50!bg, fg=black}
}



\title{The Singular Value Decomposition (SVD)}
\date{}

\begin{document}
\DeclareFontShape{OT1}{cmss}{b}{n}{<->ssub * cmss/bx/n}{} 
\begin{frame}
	\titlepage
\end{frame}


%%%%%%%%%%%%%%%%%%%%%
%%%%%%%%%%%%%%%%%%%%%
%%%%%%%%%%%%%%%%%%%%%
%%%%%%%%%%%%%%%%%%%%%


\begin{frame}{Matrix factorisations (continued)}
The singular value decomposition (known mostly by its acronym, SVD) is yet another type of factorisation/decomposition..
\end{frame}


\begin{frame}{Singular values}
\begin{definition}[Singular value]
Let $A\in\M_{mn}(\IR)$. The \textbf{singular values} of $A$ are the real numbers 
\[
\sigma_1\geq \sigma_2\geq\cdots\sigma_n\geq 0
\]
that are the square roots of the eigenvalues of $A^TA$
\end{definition}
\end{frame}


\begin{frame}{Singular values are real and nonnegative?}
Recall that $\forall A\in\M_{mn}$, $A^TA$ is symmetric
\vfill
\textbf{Claim 1.} Real symmetric matrices have real eigenvalues
\vfill
\textbf{Proof.} $A\in\M_n(\IR)$ symmetric and $(\lambda,\bv)$ eigenpair of $A$, i.e, $A\bv=\lambda\bv$. Taking the complex conjugate, $\overline{A\bv}=\overline{\lambda\bv}$
\vfill
Since $A\in\M_n(\IR)$, $\overline{A}=A$\qquad ($z=\bar z\iff z\in\IR$)
\vfill
So
\[
A\bar\bv=\overline{A}\bar\bv=\overline{A\bv}=\overline{\lambda\bv}=\overline{\lambda}\bar\bv
\]
i.e., if $(\lambda,\bv)$ eigenpair, $(\bar\lambda,\bar\bv)$ also eigenpair
\end{frame}

\begin{frame}
Still assuming $A\in\M_n(\IR)$ symmetric and $(\lambda,\bv)$ eigenpair of $A$ and using what we just proved (that $(\bar\lambda,\bar\bv)$ also eigenpair), take transposes
\begin{align*}
A\bar\bv = \bar\lambda\bar\bv &\iff (A\bar\bv)^T = (\bar\lambda\bar\bv)^T \\
&\iff \bar\bv^TA^T=\bar\lambda\bar\bv^T \\
&\iff \bar\bv^T A = \bar\lambda\bar\bv^T \qquad{\textrm{[$A$ symmetric]}}
\end{align*}
\vfill
Let us now compute $\lambda (\bar\bv\bullet\bv)$. We have
\begin{align*}
\lambda (\bar\bv\bullet\bv) &= \lambda\bar\bv^T\bv = \bar\bv^T(\lambda\bv) \\
&= \bar\bv^T(A\bv) = (\bar\bv^TA)\bv \\
&= (\bar\lambda\bar\bv^T)\bv = \bar\lambda(\bar\bv\bullet\bv) \\
&\iff (\lambda-\bar\lambda)(\bar\bv\bullet\bv) = 0
\end{align*}
\end{frame}

\begin{frame}
We have shown
\[
(\lambda-\bar\lambda)(\bar\bv\bullet\bv) = 0
\]
Let 
\[
\bv = \begin{pmatrix}
a_1+ib_1 \\
\vdots \\
a_n+ib_n
\end{pmatrix}
\]
Then
\[
\bar\bv = \begin{pmatrix}
a_1-ib_1 \\
\vdots \\
a_n-ib_n
\end{pmatrix}
\]
So
\[
\bar\bv\bullet\bv = (a_1^2+b_1^2)+\cdots+(a_n^2+b_n^2)
\]
But $\bv$ eigenvector is $\neq\b0$, so $\bar\bv\bullet\bv\neq 0$, so
\[
(\lambda-\bar\lambda)(\bar\bv\bullet\bv) = 0
\iff \lambda-\bar\lambda=0
\iff \lambda=\bar\lambda\iff \lambda\in\IR
\]
\end{frame}


\begin{frame}
\textbf{Claim 2.} For $A\in\M_{mn}(\IR)$, the eigenvalues of $A^TA$ are real and nonnegative

\vfill
\textbf{Proof.}
We know that for $A\in\M_{mn}$, $A^TA$ symmetric and from previous claim, if $A\in\M_{mn}(\IR)$, then $A^TA$ is symmetric and real and with real eigenvalues
\vfill
Let $(\lambda,\bv)$ be an eigenpair of $A^TA$, with $\bv$ chosen so that $\|\bv\|=1$
\vfill 
Norms are functions $V\to\IR_+$, so $\|A\bv\|$ and $\|A\bv\|^2$ are $\geq 0$ and thus
\begin{align*}
0\leq \|A\bv\|^2 &= (A\bv)\bullet(A\bv) = (A\bv)^T(A\bv) \\
&= \bv^TA^TA\bv = \bv^T(A^TA\bv) = \bv^T(\lambda\bv) \\
&= \lambda(\bv^T\bv) = \lambda(\bv\bullet\bv) = \lambda\|\bv\|^2 \\
&= \lambda
\end{align*}
\end{frame}


\begin{frame}{The singular value decomposition (SVD)}
\begin{importanttheorem}[SVD]\label{th:SVD}
$A\in\M_{mn}$ with singular values $\sigma_1\geq\cdots\geq\sigma_r>0$ and $\sigma_{r+1}=\cdots=\sigma_n=0$
\vskip0.5cm
Then there exists $U\in\M_m$ orthogonal, $V\in\M_n$ orthogonal and a block matrix $\Sigma\in\M_{mn}$ taking the form
\[
\Sigma=
\begin{pmatrix}
D & 0_{r,n-r} \\
0_{m-r,r} & 0_{m-r,n-r}
\end{pmatrix}
\]
where 
\[
D = \mathsf{diag}(\sigma_1,\ldots,\sigma_r)\in\M_r
\] 
such that
\[
A=U\Sigma V^T
\]
\end{importanttheorem}
\end{frame}


\begin{frame}
\begin{definition}
We call a factorisation as in Theorem~\ref{th:SVD} the \textbf{singular value decomposition} of $A$. The columns of $U$ and $V$ are, respectively, the \textbf{left} and \textbf{right singular vectors} of $A$
\end{definition}
$U$ and $V^T$ are \emph{rotation} or \emph{reflection} matrices, $\Sigma$ is a \emph{scaling} matrix
\end{frame}


\begin{frame}{Outer product form of the SVD}
\begin{theorem}[Outer product form of the SVD]\label{th:SVD_outer_product_form}
$A\in\M_{mn}$ with singular values $\sigma_1\geq\cdots\geq\sigma_r>0$ and $\sigma_{r+1}=\cdots=\sigma_n=0$, $\bu_1,\ldots,\bu_r$ and $\bv_1,\ldots,\bv_r$, respectively, left and right singular vectors of $A$ corresponding to these singular values
\vskip0.5cm
Then 
\[
A=\sigma_1\bu_1\bv_1^T+\cdots+\sigma_r\bu_r\bv_r^T
\]
\end{theorem}
\end{frame}


\begin{frame}{Computing the SVD (case of $\neq$ eigenvalues)}
To compute the SVD, we use the following result
\vfill
\begin{theorem}\label{th:eigenvectors_of_symmetric_are_orthogonal}
Let $A\in\M_n$ symmetric, $(\lambda_1,\bu_1)$ and $(\lambda_2,\bu_2)$ be eigenpairs, $\lambda_1\neq\lambda_2$. Then $\bu_1\bullet\bu_2=0$
\end{theorem}
\end{frame}

\begin{frame}{Proof of Theorem~\ref{th:eigenvectors_of_symmetric_are_orthogonal}}
$A\in\M_n$ symmetric, $(\lambda_1,\bu_1)$ and $(\lambda_2,\bu_2)$ eigenpairs with $\lambda_1\neq\lambda_2$
\begin{align*}
\lambda_1(\bv_1\bullet\bv_2) 
&= (\lambda_1\bv_1)\bullet\bv_2 \\
&= A\bv_1\bullet\bv_2 \\
&= (A\bv_1)^T\bv_2 \\
&= \bv_1^TA^T\bv_2 \\
&= \bv_1^T(A\bv_2)  \qquad\textrm{[$A$ symmetric so $A^T=A$]} \\
&= \bv_1^T(\lambda_2\bv_2) \\
&= \lambda_2(\bv_1^T\bv_2) \\
&= \lambda_2(\bv_1\bullet\bv_2)
\end{align*}
\vfill
So $(\lambda_1-\lambda_2)(\bv_1\bullet\bv_2)=0$. But $\lambda_1\neq\lambda_2$, so $\bv_1\bullet\bv_2=0$
\end{frame}


\begin{frame}{Computing the SVD (case of $\neq$ eigenvalues)}
If all eigenvalues of $A^TA$ are distinct, we can use Theorem~\ref{th:eigenvectors_of_symmetric_are_orthogonal}
\vfill
\begin{enumerate}
\item Compute $A^TA\in\M_n$
\item Compute eigenvalues $\lambda_1,\ldots,\lambda_n$ of $A^TA$; order them as $\lambda_1>\cdots>\lambda_n\geq 0$ ($>$ not $\geq$ since $\neq$)
\item Compute singular values $\sigma_1=\sqrt{\lambda_1},\ldots,\sigma_n=\sqrt{\lambda_n}$
\item Diagonal matrix $D$ in $\Sigma$ is either in $\M_n$ (if $\sigma_n>0$) or in $\M_{n-1}$ (if $\sigma_n=0$)
\end{enumerate}
\end{frame}


\begin{frame}
\begin{enumerate}
\setcounter{enumi}{4}
\item Since eigenvalues are distinct, Theorem~\ref{th:eigenvectors_of_symmetric_are_orthogonal} $\implies$ eigenvectors are orthogonal set. Compute these eigenvectors in the same order as the eigenvalues
\item Normalise them and use them to make the matrix $V$, i.e., $V=[\bv_1\cdots\bv_n]$
\item To find the $\bu_i$, compute, for $i=1,\ldots,r$,
\[
\bu_i = \frac{1}{\sigma_i}A\bv_i
\]
and ensure that $\|\bu_i\|=1$
\end{enumerate}
\end{frame}


\begin{frame}{Computing the SVD (case where some eigenvalues are $=$)}
\begin{enumerate}
\item Compute $A^TA\in\M_n$
\item Compute eigenvalues $\lambda_1,\ldots,\lambda_n$ of $A^TA$; order them as $\lambda_1\geq\cdots\geq\lambda_n\geq 0$
\item Compute singular values $\sigma_1=\sqrt{\lambda_1},\ldots,\sigma_n=\sqrt{\lambda_n}$, with $r\leq n$ the index of the last positive singular value
\item For eigenvalues that are distinct, proceed as before
\item For eigenvalues with multiplicity $>1$, we need to ensure that the resulting eigenvectors are LI \emph{and} orthogonal
\end{enumerate}
\end{frame}

\begin{frame}{Dealing with eigenvalues with multiplicity $>1$}
When an eigenvalue has (algebraic) multiplicity $>1$, e.g., characteristic polynomial contains a factor like $(\lambda-2)^2$, things can become a little bit more complicated
\vfill
The proper way to deal with this involves the so-called Jordan Normal Form (another matrix decomposition)
\vfill
In short: not all square matrices are diagonalisable, but all square matrices admit a JNF
\end{frame}


\begin{frame}
Sometimes, we can find several LI eigenvectors associated to the same eigenvalue. Check this. If not, need to use the following
\vfill
\begin{definition}[Generalised eigenvectors]
$\bx\neq\b0$ \textbf{generalized eigenvector} of rank $m$ of $A\in\M_n$ corresponding to eigenvalue $\lambda$ if
\[
(A-\lambda\II)^{m}\bx = \b0
\]
but
\[
(A-\lambda\II)^{m-1}\bx\neq \b0
\]
\end{definition}
\end{frame}


\begin{frame}{Procedure for generalised eigenvectors}
$A\in\M_n$ and assume $\lambda$ eigenvalue with algebraic multiplicity $k$
\vfill
Find $\bv_1$, ``classic" eigenvector, i.e., $\bv_1\neq\b0$ s.t. $(A-\lambda\II)\bv_1=\b0$
\vfill
Find generalised eigenvector $\bv_2$ of rank 2 by solving for $\bv_2\neq\b0$,
\[
(A-\lambda\II)\bv_2 = \bv_1
\]
$\ldots$
\vfill
Find generalised eigenvector $\bv_k$ of rank $k$ by solving for $\bv_k\neq\b0$,
\[
(A-\lambda\II)\bv_k = \bv_{k-1}
\]
\vfill
Then $\{\bv_1,\ldots,\bv_k\}$ LI
\end{frame}


\begin{frame}{Back to the normal procedure}
With the LI eigenvectors $\{\bv_1,\ldots,\bv_k\}$ corresponding to $\lambda$
\vfill
Apply Gram-Schmidt to get orthogonal set
\vfill
For all eigenvalues with multiplicity $>1$, check that you either have LI eigenvectors or do what we just did
\vfill
When you are done, be back on your merry way to step 6 in the case where eigenvalues are all $\neq$
\vfill
I am caricaturing a little here: there can be cases that do not work exactly like this, but this is general enough..
\end{frame}

\begin{frame}{Applications of the SVD}
Many applications of the SVD, both theoretical and practical..
\vfill
\begin{enumerate}
\item Obtaining a unique solutions to least squares when $A^TA$ singular
\item Image compression
\end{enumerate}
\end{frame}


\begin{frame}{Least squares revisited}
\begin{theorem}
Let $A\in\M_{mn}$, $\bx\in\IR^n$ and $\bb\in\IR^m$. The least squares problem $A\bx=\bb$ has a unique least squares solution $\tilde\bx$ of \emph{minimal length} (closest to the origin) given by
\[
\tilde\bx = A^+\bb
\]
where $A^+$ is the \emph{pseudoinverse} of $A$
\end{theorem}
\end{frame}

\begin{frame}
\begin{definition}[Pseudoinverse]
$A=U\Sigma V^T$ an SVD for $A\in\M_{mn}$, where 
\[
\Sigma = \begin{pmatrix}
D & 0 \\ 0 & 0
\end{pmatrix},
\textrm{ with }
D=\mathsf{diag}(\sigma_1,\ldots,\sigma_r)
\]
($D$ contains the nonzero singular values of $A$ ordered as usual)
\vskip0.5cm
The \textbf{pseudoinverse} (or \textbf{Moore-Penrose inverse}) of $A$ is $A^+\in\M_{nm}$ given by
\[
A^+ = V\Sigma^+ U^T
\]
with
\[
\Sigma^+ =
\begin{pmatrix}
D^{-1} & 0 \\ 0 & 0
\end{pmatrix}\in\M_{nm}
\]
\end{definition}
\end{frame}


\begin{frame}{Compressing images}
Consider an image (for simplicity, assume in shades of grey). This can be stored in a matrix $A\in\M_{mn}$
\vfill
Take the SVD of $A$. Then the small singular values carry information about the regions with little variation and can perhaps be omitted, whereas the large singular values carry information about more ``dynamic'' regions of the image
\vfill
Suppose $A$  has $r$ nonzero singular values. For $k\leq r$, let
\[
A_k = \sigma_1\bu_1\bv_1^T+\cdots+\sigma_k\bu_k\bv_k^T
\]
(so for $k=r$ we get the usual outer product form)
\end{frame}

\end{document}